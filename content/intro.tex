\chapter{Introduction} \label{chapter:intro}

\section{Motivation}

The problem of good boy behavior has been well established in the literature.
We follow the convention of \citet{Neyman1923Application} for the potential outcomes model.
Moreover, we build on the work of \citet{Harshaw2021Balancing} and \citet{harshaw2021design} for design and analysis of experiments.
We really cited these just so we'd have a non-empty bibliography.

In order to ensure that we have made proper use of the \texttt{figure} directory, we should create a figure.
Figure~\ref{fig:handsome-dan} shows the optimal good boy, the original Handsome Dan.
Finally, we end this motivation with nonsense text to fill it out.

\begin{figure}
	\centering
	\includegraphics{figures/handsome-dan}
	\caption{Optimal Good Boy} \label{fig:handsome-dan}
\end{figure}

\Blindtext