\begin{center}
	Abstract\\
	Algorithmic Advances for the  \\ Design and Analysis of Being a Good Boy \\
	Handsome Dan \\
	20XX
\end{center}

\begin{spacing}{2}
	
Since the beginning of recorded history, being a good boy is one of the most fundamental societal problems facing all canines.
In this dissertation, we develop a mathematical framework for being a good boy and provide algorithms for optimal good boy behavior.

In the first chapter, we present a decision theoretic framework for modeling standard good boy tricks, including ``sit'', ``stay'', ``paw'', and ``down''.
We show our framework is flexible enough to incorporate Bayesian priors based on standard exponential families.
As an application of our framework, we demonstrate that the problem of learning to optimally walk your human is polynomial time solvable.

In the second chapter, we present a suite of algorithms for computing optimal good boy behavior.
We show that, perhaps surprisingly, greedy algorithms fail to provide even a constant factor approximation.
On the other hand, we develop a class of semidefinite programming based algorithms which allow for approximately optimal good boy behavior in polynomial time.
	 

\end{spacing}

